\documentclass[aspectratio=169, 10pt]{beamer}

% ============================================================================
% THEME & PACKAGES
% ============================================================================
\usetheme{Madrid}
\usecolortheme{whale}

\usepackage[utf8]{inputenc}
\usepackage[T1]{fontenc}
\usepackage{lmodern}
\usepackage{graphicx}
\usepackage{booktabs}
\usepackage{caption}
\usepackage{subcaption}

\graphicspath{{figures/}}

% Define custom colors if needed
\definecolor{uvtblue}{RGB}{0, 85, 150}
\setbeamercolor{structure}{fg=uvtblue}

% ============================================================================
% META INFO
% ============================================================================
\title[Comparative Study: PyMOO vs Optuna]{Comparative Study of Metaheuristic Optimization Libraries:\\PyMOO vs Optuna}
\subtitle{Metaheuristic Algorithms Course Project}
\author[Patru Gheorghe Eduard]{Patru Gheorghe Eduard}
\institute[UVT]{West University of Timișoara}
\date{February 2026}

% ============================================================================
% SLIDES
% ============================================================================
\begin{document}

% ----------------------------------------------------------------------------
% Slide 1: Title
% ----------------------------------------------------------------------------
\begin{frame}
    \titlepage
\end{frame}

% ----------------------------------------------------------------------------
% Slide 2: Overview
% ----------------------------------------------------------------------------
\begin{frame}{Overview}
    \tableofcontents
\end{frame}

% ----------------------------------------------------------------------------
% Slide 3: Introduction - What is Optimization?
% ----------------------------------------------------------------------------
\section{Introduction}
\begin{frame}{Introduction: What is Optimization?}
    \begin{columns}
        \column{0.55\textwidth}
        \textbf{Optimization = Finding the best solution}
        
        \vspace{0.3cm}
        
        \textbf{Key Terms:}
        \begin{itemize}
            \item \textbf{Objective Function}: $f(x)$ — what we minimize (e.g., error, cost)
            \item \textbf{Search Space}: All possible solutions (here: $\mathbb{R}^{10}$)
            \item \textbf{Global Minimum}: The absolute best solution
            \item \textbf{Local Minimum}: A ``trap'' — looks good nearby but not best overall
        \end{itemize}
        
        \column{0.45\textwidth}
        \begin{alertblock}{Why Metaheuristics?}
            \begin{itemize}
                \item No gradient required
                \item Handle multimodal landscapes
                \item Escape local optima
                \item Work in high dimensions
            \end{itemize}
        \end{alertblock}
        
        \vspace{0.2cm}
        
        \footnotesize{\textit{Classic methods (gradient descent) fail on complex, non-smooth functions.}}
    \end{columns}
\end{frame}

% ----------------------------------------------------------------------------
% Slide 4: Libraries Compared
% ----------------------------------------------------------------------------
\begin{frame}{Libraries Compared: PyMOO vs Optuna}
    \textbf{Goal:} Compare two popular Python optimization libraries on standard benchmark problems.
    
    \vspace{0.2cm}
    
    \begin{columns}
        \column{0.35\textwidth}
        \begin{block}{\textbf{PyMOO}}
            \begin{itemize}
                \item Evolutionary algorithms
                \item Population-based
                \item GA, DE, PSO
                \item Academic focus
            \end{itemize}
            \footnotesize{\textit{``Population-based''} = many solutions evolve together}
        \end{block}
        
        \column{0.35\textwidth}
        \begin{block}{\textbf{Optuna}}
            \begin{itemize}
                \item ML hyperparameter tuning
                \item Trial-by-trial sampling
                \item TPE, CMA-ES
                \item Industry focus
            \end{itemize}
            \footnotesize{\textit{``Sampler-based''} = one trial at a time}
        \end{block}
        
        \column{0.3\textwidth}
        \begin{alertblock}{Key Metrics}
            We evaluate:
            \begin{itemize}
                \item Solution Quality (Fitness)
                \item Convergence Speed
                \item Execution Time
            \end{itemize}
        \end{alertblock}
    \end{columns}
    
    \vspace{0.2cm}
    
    \centering
    \textbf{Research Question}: Can Optuna match PyMOO on continuous function optimization?
\end{frame}

% ----------------------------------------------------------------------------
% Slide 5: Algorithms Explained
% ----------------------------------------------------------------------------
\section{Methodology}
\begin{frame}{Algorithms: How They Work}
    \begin{table}
        \centering
        \small
        \begin{tabular}{l l l l}
            \toprule
            \textbf{PyMOO} & \textbf{Optuna} & \textbf{Core Idea} & \textbf{Key Mechanism} \\
            \midrule
            \textbf{GA} & TPE & Selection + Crossover & Fitness-based survival \\
            \textbf{DE} & CMA-ES & Vector differences & Adaptive mutation \\
            \textbf{PSO} & NSGA-II & Swarm movement & Personal + global best \\
            \bottomrule
        \end{tabular}
    \end{table}
    
    \vspace{0.3cm}
    
    \begin{columns}
        \column{0.33\textwidth}
        \centering
        \textbf{GA (Genetic Algorithm)}
        
        \footnotesize{Selection $\rightarrow$ Crossover $\rightarrow$ Mutation}
        
        \tiny{Mimics natural evolution}
        
        \column{0.33\textwidth}
        \centering
        \textbf{DE (Differential Evolution)}
        
        \footnotesize{$v = x_1 + F \cdot (x_2 - x_3)$}
        
        \tiny{Uses population differences for mutation}
        
        \column{0.33\textwidth}
        \centering
        \textbf{PSO (Particle Swarm)}
        
        \footnotesize{$v_{new} = w \cdot v + c_1 \cdot p_{best} + c_2 \cdot g_{best}$}
        
        \tiny{Particles follow best positions}
    \end{columns}
    
    \vspace{0.3cm}
    \hrule
    \vspace{0.2cm}
    
    \begin{columns}
        \column{0.33\textwidth}
        \centering
        \textbf{TPE (Tree-Parzen Estimator)}
        
        \footnotesize{$p(x|y) \rightarrow$ good vs bad regions}
        
        \tiny{Bayesian model of promising areas}
        
        \column{0.33\textwidth}
        \centering
        \textbf{CMA-ES (Covariance Matrix)}
        
        \footnotesize{$\mathcal{N}(\mu, \Sigma)$ adapts over time}
        
        \tiny{Learns search distribution shape}
        
        \column{0.33\textwidth}
        \centering
        \textbf{NSGA-II (Multi-Objective)}
        
        \footnotesize{Pareto ranking + crowding}
        
        \tiny{Designed for multi-objective, adapted here}
    \end{columns}
\end{frame}

% ----------------------------------------------------------------------------
% Slide 6: Benchmark Functions
% ----------------------------------------------------------------------------
\begin{frame}{Benchmark Functions: Why These 4?}
    \begin{table}
        \centering
        \small
        \begin{tabular}{l l l l}
            \toprule
            \textbf{Function} & \textbf{Type} & \textbf{Challenge} & \textbf{Global Min} \\
            \midrule
            \textbf{Sphere} & Unimodal & Easy baseline (single bowl) & $f(0,...,0) = 0$ \\
            \textbf{Rastrigin} & Multimodal & Many local minima (100s) & $f(0,...,0) = 0$ \\
            \textbf{Ackley} & Multimodal & Flat outer + deep center & $f(0,...,0) = 0$ \\
            \textbf{Rosenbrock} & Valley & Narrow curved valley & $f(1,...,1) = 0$ \\
            \bottomrule
        \end{tabular}
    \end{table}
    
    \vspace{0.3cm}
    
    \begin{alertblock}{Why 10 Dimensions?}
        \begin{itemize}
            \item Balances computational cost with real complexity
            \item 10D = $10^{10}$ possible regions to search (high-dimensional challenge)
            \item Standard in CEC benchmark competitions
        \end{itemize}
    \end{alertblock}
    
    \vspace{0.2cm}
    \centering
    \footnotesize{\textbf{Setup}: Population = 50, Generations = 100, Independent Runs = 10}
\end{frame}

% ----------------------------------------------------------------------------
% Slide 7: Results - Overall
% ----------------------------------------------------------------------------
\section{Results}
\begin{frame}{Results: Solution Quality Comparison}
    \begin{figure}
        \centering
        \includegraphics[width=0.75\textwidth]{comparison_bars.png}
        \caption{Best Fitness Achieved (Log Scale — lower is better)}
    \end{figure}
    \begin{alertblock}{Key Finding}
        PyMOO achieves \textbf{10x to 1000x better precision} than Optuna on all benchmark functions.
    \end{alertblock}
\end{frame}

% ----------------------------------------------------------------------------
% Slide 8: Results - Convergence
% ----------------------------------------------------------------------------
\begin{frame}{Results: Convergence Analysis}
    \begin{columns}
        \column{0.5\textwidth}
        \centering
        \textbf{Sphere Function (PSO vs NSGA-II)}
        \includegraphics[width=\textwidth]{convergence_sphere_pso.png}
        
        \footnotesize{PyMOO: Rapid descent to $10^{-10}$}
        
        \footnotesize{Optuna: Plateaus early at $10^{-2}$}
        
        \column{0.5\textwidth}
        \centering
        \textbf{Rastrigin Function (GA vs TPE)}
        \includegraphics[width=\textwidth]{convergence_rastrigin_ga.png}
        
        \footnotesize{PyMOO: Escapes local minima effectively}
        
        \footnotesize{Optuna: Gets trapped in local optima}
    \end{columns}
    
    \vspace{0.3cm}
    
    \centering
    \small{\textit{Convergence} = how fitness improves over generations (iterations)}
\end{frame}

% ----------------------------------------------------------------------------
% Slide 9: Results - Time
% ----------------------------------------------------------------------------
\begin{frame}{Results: Execution Time Trade-off}
    \begin{columns}
        \column{0.55\textwidth}
        \includegraphics[width=\textwidth]{time_comparison.png}
        
        \column{0.45\textwidth}
        \textbf{Time Analysis:}
        
        \vspace{0.3cm}
        
        \begin{itemize}
            \item \textbf{Optuna}: 2-3x faster
            \item Lightweight: 1 trial at a time
            \item No population management overhead
        \end{itemize}
        
        \vspace{0.3cm}
        
        \begin{itemize}
            \item \textbf{PyMOO}: Slower but precise
            \item Manages 50 individuals per generation
            \item Full genetic operators each iteration
        \end{itemize}
        
        \vspace{0.3cm}
        
        \begin{exampleblock}{Trade-off}
            Speed vs Quality — PyMOO sacrifices time for precision
        \end{exampleblock}
    \end{columns}
\end{frame}

% ----------------------------------------------------------------------------
% Slide 10: Why PyMOO Wins
% ----------------------------------------------------------------------------
\section{Discussion}
\begin{frame}{Discussion: Why Does PyMOO Win?}
    \begin{columns}
        \column{0.5\textwidth}
        \begin{block}{\textbf{PyMOO Advantages}}
            \begin{enumerate}
                \item \textbf{True population diversity}
                
                \footnotesize{50 solutions explore simultaneously}
                
                \item \textbf{Selection pressure}
                
                \footnotesize{Best individuals survive and reproduce}
                
                \item \textbf{Genetic operators}
                
                \footnotesize{Crossover combines good traits; Mutation explores new areas}
                
                \item \textbf{Built for this task}
                
                \footnotesize{Designed specifically for continuous optimization}
            \end{enumerate}
        \end{block}
        
        \column{0.5\textwidth}
        \begin{alertblock}{\textbf{Optuna Limitations (for this task)}}
            \begin{enumerate}
                \item \textbf{Sequential sampling}
                
                \footnotesize{Only 1 trial at a time, no population}
                
                \item \textbf{TPE is Bayesian}
                
                \footnotesize{Good for expensive evaluations, not for cheap functions}
                
                \item \textbf{Designed for ML}
                
                \footnotesize{Categorical/integer params, not continuous optimization}
            \end{enumerate}
        \end{alertblock}
    \end{columns}
\end{frame}

% ----------------------------------------------------------------------------
% Slide 11: When to Use Each
% ----------------------------------------------------------------------------
\begin{frame}{Practical Recommendation: When to Use Each?}
    \begin{columns}
        \column{0.5\textwidth}
        \begin{exampleblock}{\textbf{Use PyMOO When:}}
            \begin{itemize}
                \item Continuous function optimization
                \item Mathematical benchmarks
                \item Multi-objective problems
                \item Academic research
                \item Precision is critical
            \end{itemize}
        \end{exampleblock}
        
        \column{0.5\textwidth}
        \begin{exampleblock}{\textbf{Use Optuna When:}}
            \begin{itemize}
                \item ML hyperparameter tuning
                \item Mixed parameter types (int, cat, float)
                \item Expensive objective functions
                \item Early stopping / pruning needed
                \item Distributed training
            \end{itemize}
        \end{exampleblock}
    \end{columns}
    
    \vspace{0.5cm}
    
    \centering
    \large{\textbf{Key Insight}: Right tool for the right job!}
\end{frame}

% ----------------------------------------------------------------------------
% Slide 12: Statistical Validity
% ----------------------------------------------------------------------------
\begin{frame}{Statistical Validity: Why 10 Runs?}
    \begin{columns}
        \column{0.6\textwidth}
        \textbf{Metaheuristics are stochastic} (random-based)
        
        \vspace{0.3cm}
        
        \begin{itemize}
            \item Each run uses different random seed (42, 43, ..., 51)
            \item Results vary between runs
            \item We report: \textbf{Mean $\pm$ Standard Deviation}
        \end{itemize}
        
        \vspace{0.3cm}
        
        \textbf{What Std Dev tells us:}
        \begin{itemize}
            \item Low Std Dev = Algorithm is \textbf{reliable/stable}
            \item High Std Dev = Results are \textbf{inconsistent}
        \end{itemize}
        
        \column{0.4\textwidth}
        \begin{block}{Fair Comparison}
            \begin{itemize}
                \item Same population size (50)
                \item Same generations (100)
                \item Same function evaluations
                \item Same random seeds
            \end{itemize}
        \end{block}
        
        \vspace{0.2cm}
        
        \footnotesize{\textit{This ensures reproducibility and fair comparison between libraries.}}
    \end{columns}
\end{frame}

% ----------------------------------------------------------------------------
% Slide 13: Conclusion
% ----------------------------------------------------------------------------
\section{Conclusion}
\begin{frame}{Conclusion}
    \large
    \begin{enumerate}
        \setlength\itemsep{0.8em}
        \item \textbf{Solution Quality}: PyMOO achieves 10-1000x better fitness values
        \item \textbf{Convergence}: PyMOO converges deeper; Optuna plateaus early
        \item \textbf{Speed}: Optuna is faster (2-3x) but sacrifices precision
        \item \textbf{Design Philosophy}: Different tools for different purposes
    \end{enumerate}
    
    \vspace{0.8cm}
    
    \centering
    \begin{alertblock}{Final Verdict}
        \textbf{PyMOO} for continuous optimization research \\
        \textbf{Optuna} for ML hyperparameter tuning
    \end{alertblock}
\end{frame}

% ----------------------------------------------------------------------------
% Slide 14: Thank You
% ----------------------------------------------------------------------------
\begin{frame}
    \centering
    \Huge \textbf{Thank You!}
    
    \vspace{0.8cm}
    
    \large Questions?
    
    \vspace{1cm}
    
    \normalsize
    \begin{tabular}{l l}
        \textbf{Author}: & Patru Gheorghe Eduard \\
        \textbf{Course}: & Metaheuristic Algorithms \\
        \textbf{University}: & West University of Timișoara \\
    \end{tabular}
\end{frame}

\end{document}
