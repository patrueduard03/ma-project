\documentclass[aspectratio=169, 10pt]{beamer}

% ============================================================================
% THEME & PACKAGES
% ============================================================================
\usetheme{Madrid}
\usecolortheme{whale}

\usepackage[utf8]{inputenc}
\usepackage[T1]{fontenc}
\usepackage{lmodern}
\usepackage{graphicx}
\usepackage{booktabs}
\usepackage{caption}
\usepackage{subcaption}

\graphicspath{{figures/}}

% Define custom colors if needed
\definecolor{uvtblue}{RGB}{0, 85, 150}
\setbeamercolor{structure}{fg=uvtblue}

% ============================================================================
% META INFO
% ============================================================================
\title[Comparative Study: PyMOO vs Optuna]{Comparative Study of Metaheuristic Optimization Libraries:\\PyMOO vs Optuna}
\subtitle{Metaheuristic Algorithms Course Project}
\author[Team Name]{Metaheuristic Algorithms Course\\Master's Program in Computer Science}
\institute[UVT]{West University of Timișoara}
\date{January 2026}

% ============================================================================
% SLIDES
% ============================================================================
\begin{document}

% ----------------------------------------------------------------------------
% Slide 1: Title
% ----------------------------------------------------------------------------
\begin{frame}
    \titlepage
\end{frame}

% ----------------------------------------------------------------------------
% Slide 2: Overview
% ----------------------------------------------------------------------------
\begin{frame}{Overview}
    \tableofcontents
\end{frame}

% ----------------------------------------------------------------------------
% Slide 3: Introduction
% ----------------------------------------------------------------------------
\section{Introduction}
\begin{frame}{Introduction & Objectives}
    \begin{columns}
        \column{0.6\textwidth}
        \textbf{Goal:} Compare two popular Python optimization libraries on standard benchmark problems.
        
        \vspace{0.5cm}
        
        \textbf{1. PyMOO (Multi-Objective Optimization)}
        \begin{itemize}
            \item Designed for evolutionary computation.
            \item Population-based architecture.
            \item High transparency and academic rigor.
        \end{itemize}
        
        \vspace{0.2cm}
        
        \textbf{2. Optuna (Hyperparameter Optimization)}
        \begin{itemize}
            \item Designed for ML hyperparameter tuning.
            \item Define-by-run paradigm.
            \item Sampler-based architecture (TPE, CMA-ES).
        \end{itemize}
        
        \column{0.4\textwidth}
        \centering
        \begin{block}{Key Metric}
            We evaluate:
            \begin{itemize}
                \item Solution Quality (Fitness)
                \item Convergence Speed
                \item Execution Time
            \end{itemize}
        \end{block}
    \end{columns}
\end{frame}

% ----------------------------------------------------------------------------
% Slide 4: Methodology
% ----------------------------------------------------------------------------
\section{Methodology}
\begin{frame}{Methodology: Algorithms \& Benchmarks}
    \textbf{Algorithm Comparison Pairs:}
    \begin{table}
        \centering
        \begin{tabular}{l l l}
            \toprule
            \textbf{Category} & \textbf{PyMOO Algorithm} & \textbf{Optuna Sampler} \\
            \midrule
            Population-based & Genetic Algorithm (GA) & TPE (Tree-structured Parzen) \\
            Adaptive & Differential Evolution (DE) & CMA-ES \\
            Swarm & PSO & NSGA-II Sampler \\
            \bottomrule
        \end{tabular}
    \end{table}

    \vspace{0.2cm}
    
    \textbf{Benchmark Functions (10D):}
    \begin{itemize}
        \item \textbf{Sphere}: Unimodal, convex (Baseline).
        \item \textbf{Rastrigin}: Highly multimodal (Local optima trap).
        \item \textbf{Ackley}: Flat outer region, deep center hole.
        \item \textbf{Rosenbrock}: Valley-shaped, hard convergence.
    \end{itemize}
    
    \footnotesize{\textit{Setup: Pop=50, Gen=50, Runs=5}}
\end{frame}

% ----------------------------------------------------------------------------
% Slide 5: Results - Overall
% ----------------------------------------------------------------------------
\section{Results}
\begin{frame}{Overall Performance Comparison}
    \begin{figure}
        \centering
        \includegraphics[width=0.85\textwidth]{comparison_bars.png}
        \caption{Comparison of Best Fitness (Log Scale). Lower is better.}
    \end{figure}
    \begin{alertblock}{Observation}
        PyMOO consistently achieves orders of magnitude better precision on continuous function optimization tasks.
    \end{alertblock}
\end{frame}

% ----------------------------------------------------------------------------
% Slide 6: Results - Convergence
% ----------------------------------------------------------------------------
\begin{frame}{Convergence Analysis}
    \begin{columns}
        \column{0.5\textwidth}
        \centering
        \textbf{Sphere Function (PSO/NSGA-II)}
        \includegraphics[width=\textwidth]{convergence_sphere_pso.png}
        
        \small{PyMOO (Blue) converges rapidly.\\Optuna (Red) struggles to refine.}
        
        \column{0.5\textwidth}
        \centering
        \textbf{Rastrigin Function (GA/TPE)}
        \includegraphics[width=\textwidth]{convergence_rastrigin_ga.png}
        
        \small{PyMOO maintains population diversity.\\Optuna TPE explores broadly but shallowly.}
    \end{columns}
\end{frame}

% ----------------------------------------------------------------------------
% Slide 7: Results - Time
% ----------------------------------------------------------------------------
\begin{frame}{Execution Time Comparison}
    \begin{columns}
        \column{0.6\textwidth}
        \includegraphics[width=\textwidth]{time_comparison.png}
        
        \column{0.4\textwidth}
        \textbf{Performance Trade-off:}
        
        \vspace{0.5cm}
        
        \begin{itemize}
            \item \textbf{Optuna} is approx. \textbf{2-3x faster}.
            \item Lightweight overhead optimized for ML distinct trials.
            \item \textbf{PyMOO} has higher overhead due to full population management.
        \end{itemize}
    \end{columns}
\end{frame}

% ----------------------------------------------------------------------------
% Slide 8: Discussion
% ----------------------------------------------------------------------------
\section{Discussion}
\begin{frame}{Discussion}
    \begin{columns}
        \column{0.5\textwidth}
        \begin{block}{Why PyMOO wins on Quality?}
            \begin{itemize}
                \item True population-based selection pressure.
                \item Genetic operators (Crossover/Mutation) effectively exploit local gradients.
                \item Designed specifically for mathematical optimization.
            \end{itemize}
        \end{block}
        
        \column{0.5\textwidth}
        \begin{block}{When to use Optuna?}
            \begin{itemize}
                \item Hyperparameter tuning (Categorical/Integer parameters).
                \item Expensive objective functions (Pruning allowed).
                \item Distributed computing requirements.
            \end{itemize}
        \end{block}
    \end{columns}
\end{frame}

% ----------------------------------------------------------------------------
% Slide 9: Conclusion
% ----------------------------------------------------------------------------
\section{Conclusion}
\begin{frame}{Conclusion}
    \large
    \begin{itemize}
        \setlength\itemsep{1em}
        \item \textbf{Solution Quality}: PyMOO is superior for continuous benchmark problems.
        \item \textbf{Speed}: Optuna is faster but less precise for this domain.
        \item \textbf{Usability}: Both libraries have excellent APIs, but serve different core purposes.
    \end{itemize}
    
    \vspace{1cm}
    
    \centering
    \textbf{Final Verdict: Use PyMOO for algorithmic research, Optuna for system tuning.}
\end{frame}

% ----------------------------------------------------------------------------
% Slide 10: Thank You
% ----------------------------------------------------------------------------
\begin{frame}
    \centering
    \Huge \textbf{Thank You!}
    
    \vspace{1cm}
    
    \large Questions?
\end{frame}

\end{document}
